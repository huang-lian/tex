\documentclass{ctexart}
\usepackage{amsmath}
\usepackage{amsfonts}

\everymath{\displaystyle}

\begin{document}
给定\(B\),对于指定的\(C_i\),有
\begin{align}
    \frac{r_i}{\sin{\angle OAC_i}} = \frac{R}{\sin\angle AC_iO}\label{eqCi1} \\
    \frac{r_i}{\sin{\angle OBC_i}} = \frac{R}{\sin\angle BC_iO}\label{eqCi2}
\end{align}

(1) $ \tilde{C_i}\in\overset{\frown}{AB} $时

插图

\begin{align*}
    \sin \angle OAC_i & =\sin(\pi - (\angle AC_iO + \theta_i)            )      \\
                 & =\sin(\angle AC_iO + \tilde{\theta_i}+\delta_i)\\
                 & =\sin(\angle AC_iO + i\Theta +\delta_i)\\
                 &=\sin(\angle AC_iO + i\Theta)\cos\delta_i + \cos(\angle AC_iO + i\Theta)\sin(\delta_i)\\
                 &\approx \sin(\angle AC_iO + i\Theta)(1-\frac{1}{2}x^2) + \cos(\angle AC_iO + i\Theta)x
\end{align*}

\begin{align*}
    \sin\angle OBC_i & =\sin(\pi - (\angle BC_iO + \angle AOB - \theta_i))\\
                 & = \sin (\angle BC_iO + k\Theta - (\tilde{\theta_i}+\delta_i))\\
                 & = \sin (\angle BC_iO + k\Theta - i\Theta - \delta_i))\\
                 & =  \sin (\angle BC_iO + (k - i)\Theta)\cos \delta_i-\cos  (\angle BC_iO + \Theta(k - i))\sin  \delta_i\\
                 & \approx \sin (\angle BC_iO + (k - i)\Theta)(1-\frac{1}{2}x^2)-\cos  (\angle BC_iO + (k - i)\Theta)x
\end{align*}

记
\begin{align*}
    \sin(\angle AC_iO + i\Theta) &=\sin(\angle AC_iO + i\Theta)\\
    b_1 &= \cos(\angle AC_iO + i\Theta)\\
    c_1 &= \sin\angle AC_iO\\
    a_2 &= \sin (\angle BC_iO + (k - i)\Theta)\\
    b_2 &= \cos  (\angle BC_iO + (k - i)\Theta)\\
    c_2 &= \sin\angle BC_iO\\
\end{align*}

\begin{align*}
0&=(a_1c_2 - a_2c_1)x^2-2(b_1c_2+b_2c_1)x -2(a_1c_2 - a_2c_1)\\
a &= (a_1c_2 - a_2c_1)\\
&= \sin(\angle AC_iO + i\Theta)\sin\angle BC_iO - \sin (\angle BC_iO + (k - i)\Theta)\sin\angle AC_iO\\
b &= 2(b_1c_2+b_2c_1) \\
& = 2(\cos(\angle AC_iO + i\Theta)\sin\angle BC_iO+\cos  (\angle BC_iO + (k - i)\Theta)\sin\angle AC_iO) \\
c &= -2(a_1c_2 - a_2c_1)\\
 &= -2a
\end{align*}

于是方程 (\ref{eqCi1}) 和 (\ref{eqCi2}) 联立,代换后可以得到


\end{document}